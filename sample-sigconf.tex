%%
%% This is file `sample-sigconf.tex',
%% generated with the docstrip utility.
%%
%% The original source files were:
%%
%% samples.dtx  (with options: `sigconf')
%% 
%% IMPORTANT NOTICE:
%% 
%% For the copyright see the source file.
%% 
%% Any modified versions of this file must be renamed
%% with new filenames distinct from sample-sigconf.tex.
%% 
%% For distribution of the original source see the terms
%% for copying and modification in the file samples.dtx.
%% 
%% This generated file may be distributed as long as the
%% original source files, as listed above, are part of the
%% same distribution. (The sources need not necessarily be
%% in the same archive or directory.)
%%
%% The first command in your LaTeX source must be the \documentclass command.

\documentclass[sigconf, anonymous]{acmart}
\usepackage{multirow}

\usepackage{array}
\newcolumntype{C}[1]{>{\centering\arraybackslash}p{#1 em}}
% Make column the same width
\sloppy

%%
%% \BibTeX command to typeset BibTeX logo in the docs
\AtBeginDocument{%
  \providecommand\BibTeX{{%
    \normalfont B\kern-0.5em{\scshape i\kern-0.25em b}\kern-0.8em\TeX}}}

%% Rights management information.  This information is sent to you
%% when you complete the rights form.  These commands have SAMPLE
%% values in them; it is your responsibility as an author to replace
%% the commands and values with those provided to you when you
%% complete the rights form.
\setcopyright{acmcopyright}
\copyrightyear{2021}
\acmYear{2021}
\acmDOI{10.1145/1122445.1122456}

%% These commands are for a PROCEEDINGS abstract or paper.
\acmConference[WWW '21]{WWW '21: The Web Conference 2021}{April 19--23, 2021}{Ljubljana, Slovenia}
\acmBooktitle{WWW '21: The Web Conference 2021,
April 19--23, 2021, Ljubljana, Slovenia}
\acmPrice{}
\acmISBN{978-1-4503-XXXX-X/18/06}


%%
%% Submission ID.
%% Use this when submitting an article to a sponsored event. You'll
%% receive a unique submission ID from the organizers
%% of the event, and this ID should be used as the parameter to this command.
%%\acmSubmissionID{123-A56-BU3}

%%
%% The majority of ACM publications use numbered citations and
%% references.  The command \citestyle{authoryear} switches to the
%% "author year" style.
%%
%% If you are preparing content for an event
%% sponsored by ACM SIGGRAPH, you must use the "author year" style of
%% citations and references.
%% Uncommenting
%% the next command will enable that style.
%%\citestyle{acmauthoryear}

%%
%% end of the preamble, start of the body of the document source.
\begin{document}

%%
%% The "title" command has an optional parameter,
%% allowing the author to define a "short title" to be used in page headers.
\title{Private Machine Learning by Random Transformations}

%%
%% The "author" command and its associated commands are used to define
%% the authors and their affiliations.
%% Of note is the shared affiliation of the first two authors, and the
%% "authornote" and "authornotemark" commands
%% used to denote shared contribution to the research.
\author{Zheng Fei}
\email{zfscgy2@zju.edu.cn}
\affiliation{%
  \institution{Zhejiang University}
}


%%
%% By default, the full list of authors will be used in the page
%% headers. Often, this list is too long, and will overlap
%% other information printed in the page headers. This command allows
%% the author to define a more concise list
%% of authors' names for this purpose.
\renewcommand{\shortauthors}{Zheng Fei, et al.}

%%
%% The abstract is a short summary of the work to be presented in the
%% article.
\begin{abstract}
  With the increasing demands for privacy protection, many privacy-preserving machine learning systems were proposed in recent years. However, most of them cannot be put into production due to their slow training and inference speed caused by the heavy cost of homomorphic encryption and secure multiparty computation(MPC) methods. To circumvent this, we proposed a privacy definition which is suitable for large amount of data in machine learning tasks. Based on that, we showed that random transformations like linear transformation and random permutation can well protect privacy. Merging random transformations and arithmetic sharing together, we designed a framework for private machine learning with high efficiency and low computation cost.
\end{abstract}

%%
%% The code below is generated by the tool at http://dl.acm.org/ccs.cfm.
%% Please copy and paste the code instead of the example below.
%%
\begin{CCSXML}
  <ccs2012>
     <concept>
         <concept_id>10002978.10003018.10003019</concept_id>
         <concept_desc>Security and privacy~Data anonymization and sanitization</concept_desc>
         <concept_significance>300</concept_significance>
         </concept>
     <concept>
         <concept_id>10010147.10010178</concept_id>
         <concept_desc>Computing methodologies~Artificial intelligence</concept_desc>
         <concept_significance>300</concept_significance>
         </concept>
     <concept>
         <concept_id>10010147.10010257</concept_id>
         <concept_desc>Computing methodologies~Machine learning</concept_desc>
         <concept_significance>300</concept_significance>
     </concept>
  </ccs2012>
\end{CCSXML}
  
\ccsdesc[300]{Security and privacy~Data anonymization and sanitization}
\ccsdesc[300]{Computing methodologies~Artificial intelligence}
\ccsdesc[300]{Computing methodologies~Machine learning}
%%
%% Keywords. The author(s) should pick words that accurately describe
%% the work being presented. Separate the keywords with commas.
\keywords{Privacy, Machine Learning, Secure Multiparty Computing}

%%
%% This command processes the author and affiliation and title
%% information and builds the first part of the formatted document.
\maketitle

\section{Introduction}
Machine learning has been widely used in many real-life scenarios in recent years. In most cases, training machine learning models requires a large amount of data. For example, training a neural network to determine whether two pictures belong to the same person may needs at least tens of thousands photos, and training a model to predict the possibility of credit default of someone needs tens of thousands credit records from different people. Those data are always distributed among different facilities. On one hand, governments across the world have published the laws against abuses of data in order to protect people's privacy and prevent those data from being stolen for evil uses. On the other hand, companies do not want their data being exposed to others. When they want to share data with others, it's always difficult to ensure that the other party will not store their data secretly for usages violates the contract. So to make different data owners to share their data and hence to build better machine learning models, privacy-preserving machine learning technologies must be adopted.
To achieve this, researchers have worked out many solutions, which can be generalized to following major methods. 
\begin{itemize}
    \item Cryptography-based Methods. There are two major kinds of cryptography-based methods: 
    
    1. Homomorphic Encryptions(HE). The homomorphic encryption methods allow arithmetic operations on the ciphertext. For example, the Paillier cryptosystem\cite{paillier1999} supports additions on ciphertext, and the Gentry cryptosystem supports both addition and multiplication, which is the first fully homomorphic encryption scheme. The level of security is based on the security parameter, which is usually in proportion with the key length.
    
    2. Secure Multiparty Computation(MPC). MPC methods provide ways to calculate a function while keep the inputs private. The most famous scheme is Yao's garbled circuit\cite{lindell2009gcproof}. Secret-sharing based methods are more efficient on arithmetic calculations, so they are always combined with garbled circuit methods. E.g. SecureML\cite{mohassel2017secureml}.
    \item Differential Privacy(DP)\cite{dwork2014dp}. DP is a tool to analyze privacy leakage. In order to reduce privacy leakage, noise can be added to the data.
    \item Federated Learning(FL)\cite{mcmahan17fed}. FL methods protect user data privacy by sending model to user devices, e.g. mobile phones, and send the model updates back to server.
\end{itemize}

However, those methods all have some shortcomings. HE and MPC requires lots of computation or communication,  also hard to implement. DP is easy to implement, but adding noises certainly has a huge negative impact on model performance. FL methods need user devices to do computations, and can only be applied to horizontally split data(i.e. different samples with same features).

\subsection{Our contributions}
Existing methods mostly focus on designing a method or protocol that will leak no information about the raw data in the cryptographic sense. Like using homomorphic encryption, no attackers can gain any information about the data in polynomial time w.r.t. security parameter. However, this definition is not suitable for the data in machine learning setting. What is necessary is that the data cannot be reused. So I proposed a metric to quantify the information leakage during computation, and a practical method to leverage between privacy preserving and efficiency.
In this paper, I made the following contributions:
\begin{itemize}
    \item A privacy definition that evaluates the possibility of reconstructing raw data from transformed data.
    \item We proved random transformations, i.e. random linear transformation and random permutation can well protect privacy.
    \item We designed a private machine learning framework combining random transformation and arithmetic sharing together, achieves very high efficiency in machine learning tasks.
\end{itemize}
\section{Related Work}
In this section, we will introduce earlier researches on privacy preserving machine learning, including traditional privacy notions, cryptography-based private preserving machine learning methods, differential privacy and federated learning.

\subsection{Privacy Preserving Methods}
Privacy preserving during the data analysis process has long been concerned. The k-anonymity\cite{k-anonymity} method is to perturb or hide some of the attributes which can be used to identify individual records, so-called the quasi identifiers. Beyond it, there are l-diversity\cite{l-diversity} aiming for adding diversity in a group of 'close' records, and t-closeness\cite{t-closeness} aiming for make the distribution similar for different group of records. However, as the era of big data and machine learning comes, the amount of data become enormous and the structure of data is fairly complicated, which is kind of incompatible with those previous privacy notions. For example, machine learning tasks usually do not need quasi-identifiers i.e. phone numbers, postcodes. But with huge amount of data, even without any quasi identifiers, privacy can be leaked. \cite{Narayanan2006netflixleak} shows that even a few records exposed, the attacker may be able to locate a specific person in the database. More recently, \cite{fredrikson2015inversion} shows that even trained models can leak some information about the training data. And DeepLeakage\cite{zhuligeng2019deepleakage} shows the possibility to reconstruct training samples from gradients.

\subsection{Cryptography-based Methods}
There are two major cryptographic methods for privacy preserving machine learning. One is homomorphic encryption, which allows arithemtic operations to be performed on the encrypted data. Another is secure multiparty computing, which enables multiple parties to jointly evaluate a function while keeping their inputs private.

Cryptonets\cite{bachrach2016cryptonets} first applied the somewhat fully homomorphic encryption to deep neural network. All computations are done on the encrypted data. The authors tested this model on the MNIST dataset, and achieved 99\% accuracy with a throughput about 59000 prediction per hour and a latency for about 250 seconds. Gazelle\cite{juvekar2018gazelle} avoided expensive fully homomorphic encryption and used packed additive homomorphic encryption to improve efficiency, and used garbled circuit to calculate non-linear activations. It reduces single image classification latency to around 30 microseconds. GELU-Net\cite{zhangqiao2019gelunet} let clients to calculate the activations while server calculate the linear transformation using additive homomorphic encryption. 

Alongside the homomorphic encryption methods, secure multiparty computing methods are also widely used. Beaver\cite{beaver1991} used precomputed triples to accelerate online multiplication. 
ABY\cite{demmler2015aby} mixed arithmetic, boolean and Yao sharing together provided a efficient two-party computation protocol that supports various kinds of computations covered common machine learning functions. ABY3\cite{mohassel2018aby3} hugely improved its efficiency under 3PC setting. 
\newline
SecureML\cite{mohassel2017secureml} combined arithmetic sharing and garbled circuit together to perform linear regression, logistic regression and neural network. Chameleon\cite{riazi2018chameleon} improved ABY by third-party aided multiplication triple generation and adopting GMW protocols\cite{GMW1987}. SecureNN\cite{wagh2019securenn} uses four sharing schemes and customized protocols for private comparison, outperforms previous works on four convolutional networks.

Those methods all have their advantages: Homomorphic encryption requires only one-round communication, and the security is proven by cryptography; Multiparty computing is faster in most cases, and can be more efficient with semi-honest third party introduced; However, these methods still requires heavy computation and massive communication, and are often hard to implement since they requires modern cryptographic methods, which can be difficult for machine learning developers.

\subsection{Differential Privacy}
Differential privacy was proposed by \cite{dwork2014dp}. It protects privacy by limiting the change of function when one record in the data changed. Differential privacy is always achieved by adding noises somewhere in the data analysis process. For example, \cite{abadi2016deepdp} adds the noise in the gradients of training. PATE\cite{nicolas2017pate} applied differential privacy on the label-generation phase of teacher models. And the ESA architecture\cite{Bittau2017ESA} uses local differential privacy to ensure the worst-case privacy when all other parties are colluding together.
Differential privacy provided a strong tool to evaluate privacy, but it will certainly affects the model performance in machine learning since noises are added. And in our view, it's focused on the maximum effect of one record, but not the actual sensitive data. In other words, maybe we do not need such a strict standard in order to protect data privacy. 

\subsection{Federated learning}
Federated learning was first proposed by \cite{mcmahan17fed}. It trains the model in a decentralized manner by sending model parameters to user devices, then the model is trained locally on those devices. After that, server gathers the model updates and calculates the final update. In order to prevent privacy leakage by uploading model updates, secure aggregation\cite{bonawitz2017aggre} and homomorphic encryption\cite{phong2018additive} methods were proposed. However, federated learning methods are designed for horizontally split data, i.e. different data owners have different samples, but the features are overlapped. But sometimes, data can be vertically split, i.e. same samples, different features. And let user devices to train the model will certainly affects the efficiency and performance.
\section{Privacy Definition}
In this section, we proposed a new privacy definition named reconstructive privacy. Reconstructive privacy evaluated the possibility to reconstruct raw data from transformed data. It is a powerful tool to analyze information leakage during different transformations. Based on this, we proved random permutations and linear transformations leaks very little information of raw data.
\subsection{Reconstructive Privacy}
In most machine learning scenarios, the data used is in a tabular form. Every row is a training sample and every column is a feature. The metadata, i.e. the ID of each row and the attribute name of each column are very easy to be hide. The only information exposed is the entries of the data table. 

How can an attacker use leaked data for his own malicious purpose? There should be two ways for an attacker to use the leaked data:
\begin{enumerate}
    \item Linking: The attacker have access to some other data, with overlapped samples. And then he can link the leaked data with the other data, and then gained some extra information.
    \item Reusing: Although the attacker cannot find other data to link with leaked data, he can still store the leaked data and illegally reuse them some time later.
\end{enumerate}

Thanks to cryptography, there are plenty of ways to hide sample's ID while several parties wants to jointly train some model. E.g. Private Set Intersection(PSI) methods. So the attacker cannot directly reuses leaked data. The only way is to link the leaked data to some other data, then 're-identify' the leaked data. Linking requires common columns, so the attacker must obtain the original columns of the data. In order to measure attacker's ability to recover the raw data columns from transformed data, we defined reconstructive privacy as follows:

\begin{definition}[$\epsilon$-reconstructive privacy]
    A data transformation $T$ is said to have $\epsilon$-reconstructive privacy under auxiliary information $a$ if an adversary $\mathcal A$ with input $T(x)$ and auxiliary information $a$ has a chance at most $\epsilon$ to recover the raw data $x$.
    In other words, for any function $A$, $E_x(p[A(T(x); a) = x]) < \epsilon$. If there are no auxiliary , 
\end{definition}

For example, consider shuffle on an array of length 5. Without any other information, the adversary can only guess randomly. So he has a $\dfrac1{5!}$ to get the correct raw data. That is, the transformation shuffle has a $\dfrac1{5!}$-reconstructive privacy with no auxiliary information.

\begin{definition}[$\epsilon,\delta$-reconstructive privacy]
    A data transformation $T$ is said to have $\epsilon$-reconstructive privacy under auxiliary information $a$ if an adversary $\mathcal A$ with input $T(x)$ and auxiliary information $a$ has chance $\epsilon$ to recover the raw data $x$ with error $\delta$. The error definition can be specifically chosen according to scenario.
    In other words, for any function $A$, $E_x(p[|A(T(x); a) - x| < \delta]) < \epsilon$.
\end{definition}

For example, consider adding noises $e \sim \mathcal N(0,1)$ to a value $x$. The adversary get the value $x + e$, with auxiliary input that the variation of noise is 1. so he can guess the real value is in $[x + e -3, x + e + 3]$ with a $\Psi(3) - \Psi(-3) = 0.997$ confidence. That is, the transformation $T(x) = x + e$ has a $0.997,3$-reconstructive privacy.

\subsection{Common Transformations}
\textbf{Linear Transformation: }
Let raw data be a vector  $\mathbf x$ of length $n$. A linear transformation turns $x$ into $Ax$ where $A\in \mathbb R^{m\times n}$ is a matrix. Calculating the $\epsilon$ and $\delta$ in linear transformation's reconstructive privacy is not trivial. In the following theorem, we assume that the raw data $x$ and the elements in matrix $A$ are all random variables drawn from a standard normal distribution.

\begin{theorem}[Linear transformation's reconstructive privacy]\label{thm:linear}
    Let raw data $\mathbf x \in \mathbb R^n$ be a vector with each element drawn from the standard normal distribution independently, the same as matrix $A \in \mathbb R^{m\times n}$. And let the auxiliary information for adversary is the $\mathbf x$ and $A$, both are drawn from standard normal distribution. The linear transformation $T: x\rightarrow Ax$ has a $\epsilon, \delta$-reconstructive privacy, where $\epsilon < p(|\mathbf y| <\delta)$ with $y\in \mathbb R^{n-1}, y_1,...,y_{n-1} \sim \mathcal N(0, 1)$ and $p$ is the density function. In other words, it's like the adversary has no information in $n - 1$ dimensions of the raw data $x$.
\end{theorem}

\begin{proof}
    First, from the reconstructive privacy's definition, we have 
    \begin{equation}
        \label{def:raw}
        E_\mathbf x(p[|C(A\mathbf x) - \mathbf x| < \delta]) < \epsilon
    \end{equation}
     Here we use $C(·)$ to denote adveray's function to avoid the confusion with transformation matrix $A$. Since $A$ is also a random variable, we can change \eqref{def:raw} into $E_{\mathbf x, A}I(|C(A\mathbf x) - \mathbf x| <\delta)$, where $I$ is a indicator function when the condition satisfies is 1, otherwise is 0. 
    \begin{equation}
        \epsilon =\dfrac{ \int_{A, \mathbf x}p(T=A)p(X=\mathbf x)I(|C(A\mathbf x)-\mathbf x|<\delta)d\mathbf xdA}{ \int_{A,\mathbf x}p(T=A)p(X=\mathbf x)d\mathbf xdA}
    \end{equation}
    
    Where we uses $d\mathbf x$ to denote $dx_1dx_2...dx_n$, and uses $dA$ to denote $dA_{1,1}dA_{1,2}...dA_{m, n}$. In order to eliminate the annoying term $C(Ax)$, we have to do a rotation on $\mathbf x$'s coordinates and extract $y = A\mathbf x$. That produces:
    \begin{equation}
    \begin{split}
        \dfrac{ \int_y \int_{A} \int_{\mathbf x, A\mathbf x=y}p(T=A)p(X=\mathbf x)I(|C(y)-\mathbf x|<\delta)dvdAdy}{ \int_y \int_{A} \int_{\mathbf x, A\mathbf x= y}p(T=A)p(X=\mathbf x)dvdAdy}
    \end{split}
    \end{equation}
    Then how to find the upper bound of $\epsilon$? The intuition comes from the simple inequality $\dfrac {\int f(x) dx}{\int g(x) dx} \le \max \dfrac{f(x)}{g(x)}$ for $f(x), g(x) > 0$. Considering the hyperplane $A\mathbf x = y$, the formula 
    
    $\dfrac {\int_{A\mathbf x = y}p(X=x)I(|C(y) - x| < \delta)dv}{\int_{A\mathbf x = y}p(X=x)dv}$ is actually the probability of $x$ lies in the ball $\mathcal B(y, \delta)$ on the hyperplane $A\mathbf x = y$. Since the marginal distribution on that hyperplane is still a standard normal distribution, Which can be expressed by $p(z) = \dfrac{1}{\sqrt{(2\pi)^{n-1}}}e^{-|\mathbf z|^2}$. So the upperbound of $\epsilon$ is lower than the probability that the random vector $\mathbf x \in \mathbb R^{n-1}$ has a length shorter than $\delta$.
\end{proof}

The above theorem shows the linear transformation will reveal no more information than one dimension of the raw data. Actually, since we used very strong conditions to proof the upperbound of $\epsilon$, the information leakage can be far less than theory, that is, the adversary can only get a little information on one dimension of the raw data. Notice that one dimension does not mean one element in the vector.

\textbf{Random permutation: } Random permutation is a basic method to hide data. Since a random permutation on a sequence of length $n$ can produce $n!$ possible outcomes, we can calculate the reconstructive privacy for it:
\begin{theorem}[Random permutation's reconstructive privacy]
    Random permutation on an vector of length $n$ has a 
    
    $\dfrac{1}{n!}$-reconstructive privacy.
\end{theorem}
\subsection{Attacker models}
Assume the transformed data is send to a curious third party, who tries to reconstruct the raw data with some of his own knowledge. Here we shows that the attacker can hardly gain any other information with some part of raw data or some knowledge of the transformation.
\subsubsection{Attacker with some part of raw data}
Assume raw data are a collection of samples $\{\mathbf x_1, \mathbf x_2, ... , \mathbf x_n\}$(can be also represented by $X$) with dimension $d$. And the attacker have some part of raw data $\{\mathbf x_1', \mathbf x_2' ...., \mathbf x_m'\}$ with dimension $d' < d$. In this case, the attacker's purpose is to join the table and get more attributes for his samples. The attacker have to guess the mapping from his sample to columns of transformed data $Y\in\mathbb R^{f\times n}$, where there are ${n \choose m} m!$ possibilities. 

For linear transformations, assume the transformation matrix $A\in\mathbb R^{d \times f}$ and the transformed data are $Y=XA$. And within each possibility, according to \eqref{thm:linear}, the attacker still have no more knowledge about $n-m-1$ dimensions of the raw data. Since ${n \choose m} m!$ is already big enough, and usually $n - m - 1$ should be much larger than 1, so we can say linear transformation is resilient to this sort of attack.

As for random permutation, the possibilities increases exponentially with the data size. Supposed the raw data size is $n$, and the attacker obtains $m$ raw data values, there are still $(n-m)!$ possibilities for the remaining $n - m$ values. Hence, with large enough data size, the attacker can hardly gain any knowledge on rest of the raw data.
\subsubsection{Attacker with knowledge of the transformation}
Although it's hard for attackers to know anything about the random transformation, since the data holder decides it, but it is still worth discussion. 

For random permutations, if the attacker knows some part of the permutation, say, $m$ out of $n$ entries in a permutation are revealed, then he still has no knowledge about the remaining $(n - m)$'s elements positions. Thus, he knows nothing besides what he already knows. 

And for random linear transformations $f: \mathbb R^m \rightarrow \mathbb R^n$, when $n < m$(which is in most of the cases), even if the attacker knows $f$, he cannot find $f^{-1}$ since it's not a bijective function. One $f(x)$ corresponds to infinite possible $x$. Only if the attacker also know the distribution of sample space $\{x\}$ and the samples are actually lying on some subspace of $\mathbb R^{m'}$ and $m' < n$, could he have a chance to fully reconstruct the original data. However, unlike the data, where attacker may get some of the raw data from other sources, it's impossible for attacker to know all about the transformation since it's performed locally by data owner. So this kinds of attack can be very rare. The transformation can be considered as a 'private key' of data owner, and shall be secure.

\subsection{Extending Raw Data's Definition}
In the discussion above, we refer 'raw data' as the raw data value, i.e. a vector, a table. But some times, the numeric values are not the essence of the data. For example, a online store gathered millions of people's purchase records. It uses different integers to refer different items and consumers, denoted by ID. For example, a book is represented by 1, and a T-shirt is represented by 100. So the whole dataset can be a matrix where if consumer i bought item j, the entry (i, j) is 1, otherwise 0. Randomly swaps the ID of two items or consumers for multiple times, it's impossible to reconstruct the original matrix. So does it achieves a 'exponentially  small $\epsilon$'-reconstructive privacy? Of course not. Since the real information of the dataset is not the interaction matrix. It is the user-item graph. A graph can have exponentially large number of adjacent matrices. Although two matrices may look not like each other at all, but they can be the adjacent matrices of the same graph. An attacker can get the graph, and with some background knowledge, i.e. by examining the degrees of each user node and item node, he can guess which item node corresponding to which real item. Hence he can use those data for his own benefit, i.e. training a recommender system. So are the images. Since convolutional models do the same linear transformation on different parts of the image, thus, the relations of different parts of the image is reserved, and attackers can easily guess the content of the raw image.
The above two cases show that the raw data is not always the numeric values, but some structure lies inside the numbers. So in order to achieve privacy preserving, the raw data's definition must be carefully chosen by domain experts.
\section{Framework Design}
In the last section, we showed that the random transformations can preserve privacy by disabling adversaries to recover raw data from the transformed data. So it is safe for data holders to give out its raw data to some third party to perform computation and the get back the results. To take advantage of this, we designed a framework merging random transformations and MPC operations together, in order to perform private machine learning tasks more efficiently.
\subsection{Arithmetic Sharing}
To our knowledge, arithmetic sharing was first formally proposed on the ABY\cite{demmler2015aby} framework. It's based on shamir's secret sharing scheme\cite{shamir1979share}. In this paper, we use the 2-party setting for simplicity.

\textbf{Sharing: }
A value $x$ is shared among parties $P_0, P_1$ means that $P_0$ holds a value $\langle x\rangle_0$ while $P_1$ holds a value $\langle x\rangle_1$ with the constraint $\langle x\rangle_0 +\langle x\rangle_1 = x$. To share a value $x\in \mathbb R^n$, party $P_i$ just simply pick $r \in \mathbb R^n$ and send $r$ and $x - r$ to $P_0$ and $P_1$ respectively.

Note that here we do not convert float values into fixed point integers. But it can cause problems, i.e., when $x$ is large and $r$ is small, $x - r \approx x$. In order to prevent this, we uses a $r$ whose variation is greater than $x$, so its safe to share $r$ and $x - r$. 

\textbf{Reconstruction: }
$P_0, P_1$ both send their value some party $P_i$, could be one of them or a third-party. The raw value is reconstructed immediately by $P_i$ summing two values.

\textbf{Addition: }
When adding a public value $a$ to a shared value $x$, the two parties just add $a/2$ to their shares of $x$.
When adding a shared value $a$ to a shared value $x$, the two parties just add their shares of $a$ and $x$ respectively.

\textbf{Multiplication: }
When multiplying a public value $a$, the two parties just multiply their shares by $a$.
Multiplication with a shared value is kind of tricky. We adopted the beaver triple in this framework. Suppose multiply shared value $x$ with shared value $y$. This requires a precomputed triple $uv = w$. In the sense of matrix, the $u$ should have the same shape with $x$ and $v$ should have the same shape with $y$. So $xy = (x - u)(y - v) + (x -u)v + u(y - v) + uv$. And $x - u$ and $y - v$ can be public since $u$ and $v$ are shared private values, this formula can be evaluated in a shared manner. Both parties then shares the product $xy$.

\textbf{Convolutional Layers: }
Convolutional layers are like dense layer. The filters are the weights of a dense layer, and the input image have to be transformed to a new matrix where different rows are different areas of the raw image. Let the input image be $X$ and the filters be $F$, it's easy to see the convolution operation $X*F$ satisfies the distribution law, i.e. $X*(F_1 + F_2) = X*F_1 + X*F_2$ and $(X_1 + X_2) * F = X_1*F + X_2*F$. So does the gradients (actually, the jacobian matrix) of convolution $\dfrac{\partial (X*F)}{\partial X}$ and $\dfrac{\partial (X*F)}{\partial F}$. Hence, the calculation of Convolutional layers and its gradients can be applied just like ordinary scalar or matrix multiplications.

\subsection{Adding Random Transformation to Arithmetic Sharing}
\subsubsection{Nonlinear Functions}
In previous MPC systems, the most difficult and costly part is the computation of nonlinear functions, including neural network's activation functions and logical functions, i.e. comparisons. Existing works mostly uses garbled circuit or polynomial approximation to calculate them. These methods result in heavy computation and communication costs.

By adopting reconstructive privacy, this can be quite easy. our framework contains several semi-honest third parties who can perform computation. While $P_0, P_1$ contains a shared vector $x$, when they wants to compute $f(x)$, where $f$ is some element-wise nonlinear function, they first get a random permutation of $x$, denoted by $x'$. Then they send the $x'$ to a third party $P_2$ who computes $f(x')$ and shares it to $P_0$ and $P_1$. Permuting it back, $P_1$ and $P_2$ then get the shares of $f(x)$. 

\textbf{Element-wise Functions: }
1. Either $P_0$ or $P_1$ generates a random permutation $P$ and send it to the other.
2. Each party calculates $\langle x_i'\rangle = P(\langle x_i\rangle)$ to get the permuted shared values.
2. Then they reconstruct value $x$ to a third party $P_3$. $P_3$ computes $f(x)$ then share it to $P_1$ and $P_2$. $P_1$ and $P_2$ then reconstruct the shared value using the inverse permutation $P^{-1}$ which can be easily calculated.
\subsubsection{Distribute Works to Third Party}
After appropriate random transformations, the data can be securely revealed to third party, then the third party can fit a model. But only fitting the model on transformed data, the performance will certainly dropped since the data are transformed and may lose some information. But this can be overcome by 'fitting' the transformation. Assume data $X$ is shared between two parties $P_1$ and $P_2$, so is the parameter $W_0$. Using arithmetic sharing, they can compute the matrix product $XW_0$ securely. Then they can send their shares to a third-party $P_3$ with label $Y$ and its local model $F$ with trainable parameters $W_1$. $P_3$ can then compute gradients $\dfrac{\partial L(F(XW_0), Y)}{\partial W_1}$ and $\dfrac{\partial L(F(XW_0), Y)}{\partial (XW_0)}$. .The former gradients are directly used by $P_3$ to update its local parameters. And the latter gradients are shared to $P_1$ and $P_2$. Using the chain rule, $P_1$ and $P_2$ are able to calculate their gradients $\dfrac{\partial L(F(XW_0), Y)}{\partial (W_0)}$ and then update their parameters. 

In the third-party's perspective, since he knows nothing about the $P_1$ and $P_2$'s shared parameters $W_0$, $XW_0$ can be considered as random transformations. And with the gradients sent back, the random transformations are actually learning themselves. So the whole 'shared' model can achieve same performance just like a local model.

\subsection{Put it All Together}
Combining arithmetic sharing and random transformations together, the framework mainly provides following functionalities:
\begin{itemize}
    \item Arithmetic operations, e.g. addition, subtraction, multiplication, multiplication-like operations like convolution.
    \item Element-wise functions, e.g. sigmoid, relu and their gradients.
    \item Distribute further computation to a semi-honest third party after proper transformation. E.g. for neural networks, the first layer (without activation) is performed by arithmetic sharing, then the layer outcome is reconstructed by a third party, who uses his local model to perform further computation.
\end{itemize}

This framework requires at least 1 semi-honest party to generate multiplication triples and doing element-wise function computations. Another semi-honest third party can be chosen to do further computations, or the party who holds the label data.

And the actual implementation depends on different tasks. For example, logistic regression in federated learning setting, the data holders first share their data on two semi-honest servers, with other computation service providers as helpers of matrix multiplication and performing element-wise function calculation. For deep convolutional networks, first a few layers can be performed in a secure way. After that, the output can be considered as 'random transformed', so a third party with strong computation power can do afterwards computation.

So we proposed two methods under our framework:
\begin{enumerate}
    \item RTAS: Only use random transformations for non-linear functions. This method is fit for the tasks where label holder do not have computation power, and the label(raw data, already without any ID) is sensitive. Or all parties do not want any third party to perform further training, and the feature holders do not want to reveal the transformed feature data to label holder(since the label holder have the IDs of samples, he may reuse the transformed feature data without warranty).
    \item RTAS-fast: Distrubute further computation to label holder or a third-party. This method is fit for the tasks where label is considered non-sensitive, or label holder have good computation power and the feature holders are not concerned about the reuse of transformed feature data, for example, the transformation reduced many dimensions, hence the transformed data is useless except in the current task.
\end{enumerate}
\section{Experiments}
In this section, we performed experiments on MNIST dataset using RTAS and RTAS-fast. The speed and the network traffic are measured. First experiment is logistic regression on MNIST\cite{lecun1998mnist} dataset predicting whether the given digit is 0 or not. The commercial private computing library Rosetta\footnote{https://github.com/LatticeX-Foundation/Rosetta} is used for comparation. Rosetta uses polynomial approximation for sigmoid functions, and its basic functionalities are based on SecureNN. The second experiment is a CNN for image classification on MNIST. For comparison, CryptoNets and SecureNN are used. Both training and inference speed are evaluated.
\subsection{Implementation}
To realize the framework, we uses tensorflow 2.x as the backend to perform the computations, and all computations is performed in the eager mode. We uses the GRPC library for making RPC calls across different parties. 
As for random permutation generation and inversion, we uses numpy's random generator. We creates a party to deliver all rpc calls according to the protocol, named the coordinator. When a computation needs to be performed on a party, i.e. loading a data file, doing addition, subtraction or matrix multiplication, the coordinator will generate a string representing the expression. The receiver party parses the string and do computation according to it. When the computation is finished, the receiver party saves the result tensor in its container, and then returns a unique key to the coordinator. For coordinator, the key is representing a 'remote tensor'. When one party needs the value of some other party's tensor, it also makes a rpc call. The tensor is serialized by first converting to numpy array and then uses pickle. Parallel rpc calls is made wherever it is possible. The computation can be composition of basic computations in order to reduce number of rpc calls.
\subsection{Dataset and System Settings}
We used the MNIST dataset for experiments. The MNIST dataset contains 55000 images of handwriten digits from number 0-9, equally numbered. Each image is of 8-bits gray scale and size 28 × 28. The label is a one hot vector of length 10 indicating the image belong to which number. We used 50000 images for training and 5000 for validating. The image pixel values are is scaled to [0,1) by multiply 1/256 for consistency. 

The experiment is executed on a cloud server which has 16 processor cores of frequency 2.5GHz and 64GB memory, and a Tesla T4 GPU. All the parties are simulated by individual python processes. For CryptoNets, since it only supports windows system, we implemented it on my laptop with 16Gb RAM and a 4-core intel i7-7700hq processor with base frequency 2.8GHz.
The experiments are conducted in both LAN setting and WAN setting. In LAN setting, all parties' addresses are 127.0.0.1. And in WAN setting, all parties' addresses are the public IP address of the cloud server.
\subsection{Logistic Regression}
We conducted logistic regression on MNIST dataset. The feature's dimension is 784, and the label is 'whether the digit is 0'. The internet traffic is measured by tsharks program.
The formula for logistic regression is 
\begin{equation}
    \mathbf y = sigmoid(\mathbf x W + \mathbf b) \text{ where } sigmoid(z) = \dfrac{1}{1 + e^{-z}}
\end{equation}
Since it contains only matrix addition, multiplication and element-wise function(sigmoid), it can be securely calculated by RTAS and RTAS-fast. So is its gradients.

In the experiment, the batch size is set to 32 and the learning rate is set to 0.1. Mean squared loss and SGD optimization is used.

\begin{table}[htbp]
    \caption{One-batch training/inference time for logistic regression}
    \begin{center}
    \begin{tabular}{|c|c|C{4.5}|C{4.5}|C{4.5}|}
    \hline
    Network & Model & Rosetta & RTAS & RTAS-fast \\
    \hline
    \multirow{2}{*}{LAN} & Train &  &  & \\
    \cline{2-5}
     & Infer &  &  & \\
    \hline
    \multirow{2}{*}{WAN} & Train &  &  & \\
    \cline{2-5}
     & Infer &  &  & \\
    \hline
    \end{tabular}
    \label{table:eta-log}
    \end{center}
\end{table}

\begin{table}[htbp]
    \caption{One-batch training/inference network traffic(Mb) for logistic regression}
    \begin{center}
    \begin{tabular}{|c|c|C{4.5}|C{4.5}|C{4.5}|}
    \hline
     Model & Rosetta & RTAS & RTAS-fast \\
    \hline
     Train &  &  & \\
    \hline
     Infer &  &  & \\
    \hline
    \end{tabular}
    \label{table:traffic-log}
    \end{center}
\end{table}

\subsection{Convolutional Neural Network}
We also tested a convolutional neural network(CNN) using our framework and compared the result with CryptoNets\cite{bachrach2016cryptonets} and SecureNN\cite{wagh2019securenn}. The structure of the CNN is as follows:
\begin{enumerate}
    \item Conv layer: filters=4, kernel size=4×4, stride=2, activation=relu
    \item Dense layer: input size=13*13*5, output size=100, activation=sigmoid
    \item Dense layer: input size=100, output size=10, activation=sigmoid
\end{enumerate}

For inference, we used a batch size 8192, since CryptoNets have a minimum batch size 8192. And since CryptoNets do not supports training we used batch size 32 for training.

\begin{table}[htbp]
    \small
    \caption{One-batch training/inference time(s) for CNN}
    \begin{center}
    \begin{tabular}{|c|c|c|c|c|c|}
    \hline
    Net& Model & CryptoNets & SecureNN & RTAS & RTAS-fast \\
    \hline
    \multirow{2}{*}{LAN} & Infer & 55.21 & 122.61±1.01 & 10.61±0.12  & \\
    \cline{2-6}
     & Train & - & 0.38±0.01 & 0.54±0.00 & \\
    \hline
    \multirow{2}{*}{WAN} & Infer & 55.21+100 & 6981±231 & 212.15±0.89 & \\
    \cline{2-6}
     & Train & - & 31.64±0.78 & 3.25±0.09 & \\
    \hline
    \end{tabular}
    \label{table:eta-cnn}
    \end{center}
\end{table}

\begin{table}[htbp]
    \caption{One-batch training/inference network traffic(Mb) for CNN}
    \begin{center}
    \begin{tabular}{|c|C{4.5}|C{4.5}|C{4.5}|C{4.5}|}
    \hline
     Model & Crypto-Nets & Secure-NN & RTAS & RTAS-fast \\
    \hline
     Train & - & 22632 & 333.6 & \\
    \hline
     Infer & 294 & 1052.33 & 10.23 & \\
    \hline
    \end{tabular}
    \label{table:traffic-cnn}
    \end{center}
\end{table}
\section{Conclusions}
This paper proposed a new privacy notion called reconstructive privacy. Unlike differential privacy, this privacy notion focuses on the probability of reconstructing useful information from the transformed data. And using this definition, we proved that simple random permutations and linear transformations are very hard to invert.

Based on this, those random transformations can be applied for private machine learning. Non-linear functions can then be computed easily without garbled circuits or other MPC protocols. More over, after the data transformed, third party computation servicers can perform computations on the transformed data. By comparing with Rosetta, CryptoNets and SecureNN, we showed that this method hugely reduces the computation and communication costs. 

\section{Acknowledgement}
\bibliographystyle{ACM-Reference-Format}
\bibliography{main.bib}

%%
%% If your work has an appendix, this is the place to put it.
\appendix
\end{document}
\endinput
%%
%% End of file `sample-sigconf.tex'.
